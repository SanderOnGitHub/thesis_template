% !TEX root = ../thesis.tex
\chapter[Einleitung]{Einleitung}

Eine einfache Liste:
\begin{itemize}
  \item Erstes Listenelement
  \item Zweites Listenelement
  \begin{itemize}
    \item Unterelement des zweiten Listenelements
  \end{itemize}
\end{itemize}

eine nummerierte Liste:
\begin{enumerate}%[label=\roman*]
  \item Erstes Listenelement
  \item Zweites Listenelement
  \begin{enumerate}
    \item Unterelement des zweiten Listenelements
  \end{enumerate}
\end{enumerate}

Ein Beispiel einer Liste zur Beschreibung unterschiedlicher Verfahren:
\begin{description}%[style=nextline]
  \item[Verfahren A] Beschreibung von Verfahren A.
  \item[Verfahren B] Beschreibung von Verfahren B.
\end{description}

\noindent Eine einfache Gleichung
\begin{equation}
    a^2 + b^2 = c^2
    \label{eq:example}
\end{equation}
mit dem entsprechenden Verweis auf \ref{eq:example}.

\begin{subequations} 
  \begin{align}
      \int S_\mathrm{ref}\mathrm{d}\lambda &= f_\mathrm{ko} \cdot \int S_\mathrm{pas}\mathrm{d}\lambda \label{eq:referencefactor1} \\
      \sum S_\mathrm{ref}(\lambda)\Delta\lambda &= f_\mathrm{ko} \cdot \sum S_\mathrm{pas}(\lambda)\Delta\lambda \label{eq:referencefactor2}
  \end{align}
\end{subequations}

\begin{figure}%[htb]
  \centering
  \includegraphics[scale=.3]{figures/GHS01}
  \caption{Beispiel einer Abbildung (png).}
  \label{fig:GHS01}
\end{figure}

\begin{figure}%[htb]
  \centering
  \includestandalone{figures/pas_zelle}
  \caption{Beispiel einer Abbildung (tikz).}
  \label{fig:PasZelle}
\end{figure}


\begin{figure}%[htb]
  \centering
  \includestandalone{figures/voigtlaser}
  \caption{Beispiel Spektrum der Strahlungsquelle und approximation durch Voigt Profil (tikz, pgfplots).}
  \label{fig:VoigtProfil}
\end{figure}

Eine einfache Tabelle:
\begin{table}%[htb]
  \centering
  \caption{Eine sehr einfache Tabelle. Caption über der Tabelle.}
	\label{tab:simple}
	\begin{tabular}{lcr}
        \toprule
        links & mitte & rechts \\
        \midrule
        a & b & c \\
        eins & zwei & drei \\
		\bottomrule
	\end{tabular}
\end{table}

\begin{table}%[htb]
  \centering
  \caption{Zahlen sind an der Kommastelle ausgerichtet.}
	\label{tab:simple2}
	\begin{tabular}{l*2{S[table-format=3.2]}}
        \toprule
         & l $\, /\, \si{m}$ & t $\, /\, \si{s}$ \\
        \midrule
        a & 84.92 & 5.7 \\
        a & 84.9 & 55.7 \\
		\bottomrule
	\end{tabular}
\end{table}

\begin{table}%[htb]
  \centering
  \caption{Eine etwas aufwendigere Tabelle.}
  \label{tab:anmWerte}
  \begin{tabular}{c*4{S}}
    \toprule
    & \multicolumn{2}{c}{a} & \multicolumn{2}{c}{b}\\%\multirow{2}{*}{$m$}
    \cmidrule(l){2-3}\cmidrule(l){4-5}
    $i$ & {$a_1$} & {$a_2$} & {$b_1$} & {$b_2$} \\
    \midrule
    0 & 0    & 1.21 & 2.23 & 3.23 \\
    1 & 0.58 & 1.69 & 2.71 & 3.72 \\
    2 & 0.97 & 2.13 & 3.17 & 4.19 \\ [1.2ex]
    $\Sigma$ & 1.55 & 5.03 & 8.09 & 11.14 \\
    \bottomrule
  \end{tabular}
\end{table}

\Gls{MA} wird das erste mal im Text ausgeschrieben und daraufhin mit \gls{MA} abgekürzt.
\cite{bruhns_2015,saalberg_2017,saalberg_2016}
\paragraph{}
\SI[round-precision=2]{3.2}{\micro\meter}
(\emph{Spektralbande})
